\documentclass[]{article}
\usepackage[UTF8]{ctex}
\usepackage{amsmath}
%opening
\title{《利用鸽巢定理反证Erdős–Szekeres定理》}
\author{方绍雷 416100210234}
\begin{document}
\maketitle
\thispagestyle{empty}
\clearpage
\setcounter{page}{1} % 设置当前页的开始 
\pagenumbering{arabic}
\renewcommand{\abstractname}{问题描述}
\section{问题描述}
If m and n are non-negative integers, then any sequence of
m * n+1 distinct real numbers either has an increasing subsequence of
m + 1 terms, or it has a decreasing subsequence of n + 1 terms.
\section{证明思路}
利用鸽巢定理的反证法:m,n 都为非负整数,假设m * n+1个不同实数最多含有长度为m递增序列项或者是长度n递减子序列的项。由鸽巢定理导出矛盾,证明原命题。
\section{具体证明}
对于任意不同m * n+1的实数集合。

假设序列为:$ x_{1}, x_{2},...,x_{k},....x_{m*n+1} $(m * n + 1个不同的实数)

令$ g_{k} $为序号从 1 到$ k $(包括$ k $)的递增子序列长度。

令$ f_{k} $为序号从 1 到$ k $(包括$ k $)的递减子序列长度。

对于每个实数$ x_{k} $,都有一一对应的序列对: $ (g_{k},f_{k}) $表示实数$ x_{k} $的递增子序列和递减子序列长度。

则对于每个数字,都可以得出(1)中的m * n+1个序列对。
\begin{equation}
	(g_{1},f_{1}),(g_{2},f_{2}),...,(g_{m*n+1},f_{m*n+1})
\end{equation}
而对于前提假设所知:$ 1\leq g_{k} \leq m, 1\leq f_{k} \leq n$,

根据排列组合得,一共有 $ m*n $种组合方式,而(1)中有$ m*n + 1$个序列对。由鸽巢定理所知,说明(1)中至少有两个序列对完全相同。

设相同的序列对为$ (g_{i},f_{i}),(g_{j},f_{j}) $

假设序列对重复对应数字项为$ x_{i}, x_{j}$, $ i \neq j $,由于m * n+1的实数为不同实数。
所以 $ x_{i}, x_{j}$为不同的两个实数。

易知有两种情况:
	
	当 $ x_{i} < x_{j} $时,由于  $ x_{i} < x_{j} $,得$ x_{i} $的最长递增子序列数目小于$ x_{j} $
	
	当 $ x_{i} > x_{j}$时,由于 $ x_{i} > x_{j} $,得$ x_{i} $ 的最长递减子序列数目小于$ x_{j} $
	
	所以,综上证得$ (g_{i},f_{i}),(g_{j},f_{j}) $相同的结论不成立!矛盾

所以结论m * n+1个不同实数最多含有长度为m递增序列项或者是长度n递减子序列的项不成立。
对于m * n + 1个不同实数,如果有小于m个递增子序列或者小于n个递减子序列,由鸽巢定理所得,则(1)中序列对中一定有重复。不合题意。

所以,当 $ 1\leq g_{k} \leq m + 1 $ 或者是 $ 1\leq f_{k} \leq n + 1$时,(1)中不存在重复,即由$ m * n + 1$个实数构成的不同实数序列,一定存在 m + 1个递增或者 n + 1个递减。





\end{document}
